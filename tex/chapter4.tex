\chapter{Conclusion}

In this essay I have discussed the ethical implications of autonomous vehicles transporting passengers from A to B. I have described the technology behind recent research projects developing autonomous cars, and shown an example of some recent projects, namely the Google Self-Driving Car Project. Furthermore, I have described motivations for developing autonomous cars, and the benefits they could bring.  

I have described one of many possible ethical dilemmas, and described possible way to view this ethical dilemma, depending on which ethical view is used, more precisely the utilitarianist and deontologist views. Furthermore, I have described the non-identity problem and how the topic of autonomous cars come into play with this problem.

I shown that it is not possible to determine who is to blame for autonomous cars given either an utilitarianist or deontologist view because the problems involved are not clear-cut. Furthermore I have detailed that the opinion of people regarding autonomous vehicles of other people differs from their own wishes for their own vehicles, and that this fact in combination with the issues presented earlier, show that it is difficult to find a clear-cut answer to these ethical issues. 

Finally I have concluded that the introduction of autonomous vehicles into society presents ethical dilemmas that we cannot imagine yet, and that this fact makes it very difficult to say what is right or wrong when it comes to developing the algorithms that drive the cars. 
