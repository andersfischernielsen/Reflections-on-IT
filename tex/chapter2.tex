\chapter{General Ethics and Computer Ethics}

Overall, ethics is the study of morality and moral systems. These moral systems are compromised of many components, with some common features. Bernard Gert has described the four main features of a moral system as; Public, Informal, Rational, Impartial. \cite{stanford:morality} These can be described accordingly as; The rules of the system should be know to all its members, the rules should be based on logical reason accessible to its members, there is no authority enforcing them, and the system does not treat individuals or groups differently. Furthermore, the rules of the system have connections on different levels. 

\newpar The first level of the system, which is the one that we interact with, is the rules of conduct that the system sets for us. These are rules that all members agree on, and that everyone is subjected to equally.  These can either guide the actions of people or help establish social policies. 

Rules of conduct are derived from a set of “basic moral values”, a subset of the core values of a society, which are important to its thriving and survival. The rules are subject to principals of evaluation that will justify the rules. These principals are either Religion, Law or Ethics. \cite{tavani2011ethics}

\newpar The moral system of computer ethics is based upon the systems of general ethics, and uses the same concepts and categories as base values. The values in the field of computer ethics changes alongside the evolution of the technology, however, and actions occurring brings a policy vacuum that can end in new values being formed and changes happening to existing policies. As said by Moor: 

\begin{quote}
	\blockquote{Computers provides us with new capabilities and these in turn give us new choices for action. [...] A central task of computer ethics is to determine what we should do in such cases, i.e., to formulate policies to guide our actions.}\cite{moor1985computer}
\end{quote}


\noindent Even though actions made by computers are the same as tasks previously performed by humans, tasks which already have certain morals, the way the action of a computer has been implemented can change the way the it is viewed. 

A problem of this can be described as “the invisibility factor”, which is that operations of a computer cannot be seen directly. It is therefore difficult to know if the operations are unethical. \cite{moor1985computer} Invisibility of actions in computer systems makes it difficult to evaluate with ethical principles, but it is possible to examine the design of a system, and how this design makes people act. It can be examined if the system has been designed in a way as to promotes unethical behaviour. It is possible to evaluate the morals of actions someone has taken in many ways. Therefore a perspective to look at these actions has to be chosen.

\section{Consequentialism}
In the study of normative ethics, consequentialism holds that the consequences of one's conduct are the basis of which to determine the rightness or wrongfulness of one's actions. That is, the means to which you achieve your goal pose no ethical relevance, rather the end result is what has ethical relevance. It is the idea that the end justifies the means. \cite{mizzoni2009ethics}

\subsection{Utilitarianism}
Utilitarianism is a form of consequentialism, founded by Jeremy Bentham, holding that the best moral action is the one that maximises \textit{utility}, that is the well being of sentient beings:

\begin{quote}
	\textquote{The ethical theory that holds that the action that is morally right is the one that results in the greatest possible utility (or greatest possible happiness) for the greatest number of people.} \cite{holm2013philosophy}
\end{quote}

\noindent The action that maximises the well being and minimises the suffering of humans is the one most ethically correct.

This theory is based on the assumption that all human beings strive for maximising the utility of their actions. The action that brings as much well-being as possible to as many people as possible is the morally right action. The intention of the action is not significant in utilitarianism, rather the quality of the action is “determined solely on the basis of its outcome.”, which puts utilitarianism under the class of consequentialist ethics. \cite{holm2013philosophy}

\newpar Different takes on what constitutes the utility of an action has been defined. \cite{stanford:consequentialism} Universal Consequentialism considers the happiness of all entities, and the non-well-being of a single individual therefore drowns in the well-being of all entities. Therefore a man can be killed to save others, without this being deemed "wrong". As per Mill: 

\begin{quote}
\textquote{“The individual’s concept of happiness can only be accepted insofar as it is not harmful to others.”} \cite{holm2013philosophy}	
\end{quote}


\section{Deontological Ethics}
To contrast utilitarianism, deontological theories of ethics tell us what action we ought to take. Kant states, that the only unqualifiedly good is a good will, that is the will behind an action, and therefore not the consequence of the action. \cite{stanford:deontological}

\noindent Deontology argues that actions are inherently good or bad. No matter how morally good their consequences, some choices are morally forbidden. \cite{stanford:deontological}

\newpar Entities cannot make morally wrong choices, even if by making these choices the morally wrong choices of other entities will be minimised. For deontologists, conformity with a moral norm makes a choice right. Norms are to be obeyed by each entity, but norm-keepings are not to be maximised. 

Therefore, the Right is said to have priority over the Good. If an action does not obey the Right, it may not be undertaken, no matter the Good that might come out of it. \cite{stanford:deontological}

\section{The Non-identity Problem}
The non-identity problem describes a situation where the action of bringing a person with a so-called \textit{flawed existence} into existence presents a dilemma when trying to determine whether the action is right or wrong. Bringing the person into existence usually brings a significant amount of good with it, but because the existence is flawed, the existence itself necessarily also brings some bad with it.

Three intuitions are at stake in the nonidentity problem:
\begin{itemize}
	\item The first intuition is the person-affecting, or person-based, intuition itself, that an act can only be wrong if that act makes things worse for some existing or future person.
	\item The second intuition is the intuition is that if the act gives a person an existence that is unavoidably flawed, that is the person would not exist if the act had not taken place, then the act does not make things worse for that person, because the person would otherwise not have existed.
	\item The third intuition is that some existence-inducing act are wrong, even if they do not make things worse for either the person that they bring into existence, and therefore make suffer, or any other future or existing person.
\end{itemize}


