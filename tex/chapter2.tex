\chapter{General Ethics and Computer Ethics}

Computer ethics, and generally ethics of technology, evolves with the corresponding technologies of the time. This is one of the main problems of computer ethics, and also why it is important to discuss. Moor says:

\textquote{Computers provides us with new capabilities and these in turn give us new choices for action. [...] A central task of computer ethics is to determine what we should do in such cases, i.e., to formulate policies to guide our actions.}\footnote{Quoted from \cite{moor1985computer}}

\section{Consequentialism}
In the study of normative ethics, consequentialism holds that the consequences of one's conduct are the basis of which to determine the rightness or wrongfulness of one's actions. That is, the means to which you achieve your goal pose no ethical relevance, rather the end result is what has ethical relevance. It is the idea that the end justifies the means. \cite{mizzoni2009ethics}

\textquote{Every advantage in the past is judged in the light of the final issue. — Demosthenes}

\subsection{Utilitarianism}
Utilitarianism is a form of consequentialism, founded by Jeremy Bentham, holding that the best moral action is the one that maximises \textit{utility}, that is the well being of sentient beings. The action that maximises the well being and minimises the suffering of humans is the one most ethically correct.

\textquote{The ethical theory that holds that the action that is morally right is the one that results in the greatest possible utility (or greatest possible happiness) for the greatest number of people. (Beck Holm: 207)}

\section{Deontological Ethics}
\textquote{Kant’s moral law. The point of this law is that we must always act in such a way that we can accept the consequences that would occur if all others were to act in the same way. (Beck Holm: 213)}
Deontology argues that actions are inherently good or bad. If the action performed is bad, then the entire action, no matter the outcome is bad. An action should adhere to certain moral rules, and if it does not, then it must be bad. 

\section{The Non-identity Problem}
The nonidentity problem describes a situation where bringing a person with a so-called flawed existence into existence has to be determined as being good or bad. Bringing the person into existence brings a usually significant amount of good with it, but since the existence is flawed, necessarily also some bad with it. 
Three intuitions are at stake in the nonidentity problem. The first is the person-affecting, or person-based, intuition itself, that an act can only be wrong if that act makes things worse for some existing or future person. 

The second is the intuition is that if the act gives a person an existence that is unavoidably flawed, that is the person would not exist if the act had not taken place, then the act does not make things worse for that person, because the person would otherwise not have existed.

The third intuition is that some existence-inducing act are wrong, even if they do not make things worse for either the person that they bring into existence, and therefore make suffer, or any other future or existing person. 

