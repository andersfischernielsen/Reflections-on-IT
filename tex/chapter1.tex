\chapter{Autonomous Cars}

Experiments attempting to automate vehicles, mainly cars, have been made since the 1920's with varying degrees of automation and success. 

This essay will focus on \textit{autonomous} cars, and not \textit{automated} cars. Autonomous is, as per Thesaurus, defined as: 

\begin{quote}
	\blockquote{\textit{an autonomous republic}: self-governing, independent, sovereign, free, self-ruling, self-determining, autarchic; self-sufficient.}
\end{quote}

\noindent \textit{Automation} implies that cars merely follow artificial hints in the environments, such as early experiments using magnetic strips in the road. \cite{wiki:autonomoushistory}
\textit{Autonomous} implies that cars react to their environment independently, that is they cannot depend on unnatural artefacts in their surrounding environment in order to follow the road, not crash and avoid obstacles.

\newpar Modern research has been focused on autonomous cars, since it is unrealistic to add artificial hints on every road in the world. Rather, research has been focused on making autonomous cars adapt to environments with uncertainties, so that the same autonomous vehicle can drive both in the inner city and on unpaved mountain roads. Modern vehicles furthermore have an increasing amount of automation built in, which has not brought any ethical dilemmas. On the other hand, the idea of having truly autonomous vehicles brings certain dilemmas with it, that can be researched. I will therefore not look at \textit{automatic} vehicles in this essay, but focus on \textit{autonomous} vehicles. 

\newpar Recent research projects use radar or radar-like technology in addition to GPS, odometers and computer vision in and on cars to detect the environment surrounding the car and recognise obstacles, such as people, other cars and structures, which the car will try to avoid. The addition of the radar-like LIDAR technology (a mix of the words light and radar) on the roof of the cars has provided the cars with a 200 foot-radius "view" of their surroundings, enabling them to sense the world around them in great detail. LIDAR generates a precise point cloud of the environment surrounding the car, enabling the on-board software to distinguish a running child from a bicyclist.

Google has been researching autonomous cars since 2009 \cite{googlecars}. Over the years the research project has been ongoing, the cars have developed a detailed view of their surroundings. In addition to the computer vision research the company has developed and implemented in the cars, LIDAR helps the car map its surroundings, enabling it to recognise smaller objects such as pedestrians and bicyclists. Google is using the computer vision technology it has been developing and researching to analyse the surrounding environment perceived through stereo cameras mounted on the car to detect possible collisions around the car. The sensors on the car generate 1GB of data per second\cite{datagathering}, which is analysed in order to give a precise depiction of the surrounding environment of the car. 

\section{Motivations for Developing Autonomous Vehicles}
The main motivation for developing autonomous vehicles is that human drivers are prone to make mistakes. Human drivers get distracted, have relatively slow reaction times, especially when tired or under the influence of drugs. Driving with a lack of sleep, or while under the influence of drugs, severely inhibits the driver's ability to react in time to avoid collisions. 

An autonomous car can register its surrounding environment many times a second and analyse this input to decide on the best possible action to take in a given situation. An autonomous driver does not get distracted unless programmed to do so, and has as fast a reaction time as hardware and software allows, almost guaranteed to be less than that of a human driver. An autonomous driver does not get drowsy and cannot ingest drugs. 

An autonomous will therefore, almost guaranteed, be a safer driver. 

\newpar In the United States during 2014 distracted drivers left 3179 people killed and 431000 people injured. \cite{distracteddriving}. 10\% of all crashes in the United States were crashes where the driver was identified as distracted immediately before the crash. Data on how many accidents have been caused by speeding and driving aggressively are hard to find, but account for some percentage of all crashes. 
During 2014, 31\% of all driving-related crashes were caused by impaired driving, that is driving while under the influence of alcohol or the like with 9967 people killed as a result of some of these crashes. \cite{impaireddriving}

Eliminating distracted driving and impaired driving would eliminate 41\% of all driving-related crashes, sparing the lives of 13146 people every year. Taking human error out of the equation is hoped to save many people from injury or worse.  

Furthermore, costs of human drivers can also be reduced, traffic jams might be reduced\footnote{So-called shockwave traffic-jams caused by human error, are researched in \cite{1367-2630-10-3-033001}}, and vehicles might be able to park more efficiently, and generally drive faster while still driving safely, saving time spent going from A to B. 

\newpar In short, a computer driving your car will probably be a better driver than you. An autonomous car can sense its surroundings 200 times per second and make just as many calculations reasoning for its next move based on the input. A human driver cannot top that. \cite{audirace}

This fact presents some interesting questions. Given the amount of input and processing power, an autonomous car should always be able to make the best possible choice. An autonomous car would register the child running across the road before a human driver ever could, and should therefore always make the right choice accordingly. 

\newpar But what \textit{is} the right choice? Most people would say that the car should always try to harm as few people as possible. That would be the "ethically correct thing to do". \cite{DBLP:journals/corr/BonnefonSR15} But what happens in situations where someone \textit{has} to get hurt? If an autonomous car is in a situation where it has the option of hitting two different people, but no option to avoid either one, who should it choose? What if saving both people involves killing the passengers of the car? Who should be to blame for any collisions or pay for any harm done to people or structures? 

\todo[inline]{Specificér research question.}

Answering these questions requires us to look at other ethical arguments described in the following sections.   