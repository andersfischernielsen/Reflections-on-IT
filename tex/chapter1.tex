\chapter{Autonomous Cars}

Experiments attempting to automate vehicles, mainly cars, have been made since the 1920's with varying degrees of automation and success. 

This essay will focus on \textit{autonomous} cars, not just automated cars. Autonomous is, according to Thesaurus\todo{cite}, defined as: 


\bigskip \blockquote{\textit{an autonomous republic}: self-governing, independent, sovereign, free, self-ruling, self-determining, autarchic; self-sufficient.}

\newpar \textit{Automation} implies that cars merely follow artificial hints in the environments, such as early experiments using magnetic strips in the road. 
\textit{Autonomous} implies that cars react to their environment independently, that is they cannot depend on unnatural artefacts in their surrounding environment in order to drive properly. 

Modern research has been focused on the latter, since it is unrealistic to add artificial hints on every road in the world. Rather, research has been focused on making autonomous cars adapt to environments with uncertainties, so that the same autonomous car can drive in the inner city and on mountain roads. I will therefore not look at automatic vehicles in this essay, but instead autonomous vehicles. 

Recent research projects use radar or radar-like technology in addition to GPS, odometers and computer vision in and on cars to detect the environment surrounding the car and recognise obstacles, such as people, other cars and structures, which the car will try to avoid. The addition of the radar-like LIDAR technology (a mix of the words light and radar) on the roof of the cars has provided the cars with a 200 foot-radius "view" of their surroundings, enabling them to sense the world around them in great detail. 

\section{Google's Autonomous Car Project}
Google has been researching autonomous cars since 2009 \todo{cite.}. Over the years the project has been ongoing, the cars have developed a detailed view of their surroundings. In addition to the computer vision research the company has developed and implemented in the cars, LIDAR helps the car map its surroundings, enabling it to recognise smaller objects such as pedestrians and bicyclists. 

\section{Motivations for Developing Autonomous Vehicles}
The main motivation for developing autonomous vehicles is that human drivers are prone to make mistakes. Human drivers get distracted, have relatively slow reaction times, do not always behave logically such as when angry or tired. Furthermore, human drivers sometimes drink while under the influence of drugs, which severely inhibits the driver's ability to react in time to avoid collisions. 

Labor costs of human drivers can also be eliminated, traffic jams might be reduced \todo{Cite study med random kø i alle trafik-kredse - YouTube?}, and vehicles might be able to park more efficiently, drive faster, and occupants of the car might have time to be more work while driving. 

\newpar In short, a computer driving your car will probably be a better driver than you. An autonomous car can sense its surroundings 200 times per second and make just as many calculations reasoning for its next move based on the input. A human driver will never top that. Your computer will also never get tired or drunk and make mistakes because of that. 

This fact presents some interesting questions. Given the amount of input and processing power, an autonomous car should always be able to make the best choice possible. An autonomous car would register the child running across the road before a human driver ever would, and should therefore always make the right choice accordingly. 

\newpar But what \textit{is} the right choice? Most people would say that the car should always try to harm as few people as possible. That would be the "ethically correct thing to do". But what happens in situations where someone \textit{has} to get hurt? If an autonomous car is in a situation where it has the option of hitting two different people, but no option to avoid either one, who should it choose? Who should be to blame for any collisions or pay for any harm done to people or structures? 

Answering these questions requires us to look at other ethical arguments described in the following sections.   